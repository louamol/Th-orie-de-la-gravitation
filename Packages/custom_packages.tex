\usepackage[french]{babel}
\usepackage[utf8]{inputenc}
\usepackage[T1]{fontenc}

% AJOUTS GENERAUX (base)
\usepackage{fullpage} % mise ne page
\usepackage{amsmath,amssymb,amsthm,amsfonts,amstext}
\usepackage{mathrsfs} % commande \mathscr
\usepackage{pifont} % symboles spéciaux supplémentaires
%\usepackage{fouriernc} % police de textbook
%\usepackage{lmodern} % ajoute 72 type de polices d'adobe
%\usepackage[left=2.3cm,right=2.3cm,top=3.2cm,bottom=3.2cm,footskip=1cm,marginparwidth=1.5cm]{geometry}
\usepackage{latexsym} % liste énorme de symboles
\usepackage{biblatex}
\usepackage{hyperref}
\usepackage{tikz,pgfplots} % pour les logos sur la page de garde
\usepackage{footnote}
\usepackage{color}
\usepackage{tikz-3dplot}

% AJOUTS SPECIFIQUES
\usepackage{titlesec} % permet différent styles de titres
\usepackage{csquotes} % ermet de faire des sitations
\usepackage{multirow} % permet de faire des cases  sur plusieurs colonnes
\usepackage{nicefrac} % perme de faire des fraction diagonales utilie pour les corps de texte
\usepackage{empheq} % permet de faire des systèmes déquation avec acolade et numérotation
\usepackage{physics} % ajoute des commandes pratiques pour dérivées, sin, ...
%\usepackage{bm} % meilleure commande boldsymbol
\usepackage{esvect} % \vv pour vecteurs
\usepackage{ulem} % commandes de soulignage, achurage, ... mieux
\usepackage{tensor} % permet de mieux gérer les indices des tenseurs
\usepackage{bbold}% change symbole math (entre autre R)
\usepackage[nottoc]{tocbibind}% ajoute l'appendice, ..; à la table des matièes
\usepackage[toc]{appendix}
\usepackage[font={it}]{caption}
\usepackage{subcaption}
\usepackage{enumitem}
\usepackage{xfrac}
\usepackage[procnames]{listings}
\usepackage{enumitem} % améliore l'env enumerate, ...
\usepackage{float} % permet de fixer les figures avec [H]
%\usepackage{calrsfs} % \mathcal devient très caligraphié
\usepackage{bm} % meilleur boldsymbol

% PAS ENCORE CLASSE
\usepackage{setspace}
\usepackage{graphicx}
\usepackage{CJKutf8}
\usepackage[squaren, Gray, cdot]{SIunits}
\usepackage{framed}
\usepackage{transparent}

% SETUP PACKAGES ---------------------------------
\setlength{\parindent}{0cm}
\bibliography{bibliography.bib}
\definecolor{keywords}{RGB}{255,0,90}
\definecolor{comments}{RGB}{0,0,113}
\definecolor{red}{RGB}{160,0,0}
\definecolor{green}{RGB}{0,150,0}

\theoremstyle{definition}
\theoremstyle{plain}
\newtheorem{thm}{Théorème}[chapter]
\newtheorem{prop}[thm]{Proposition}
\newtheorem{lemme}[thm]{Lemme}
\newtheorem{corol}[thm]{Corollaire}
\newtheorem{law}[thm]{Loi}
\newtheorem{propri}[thm]{Propriété}
\newtheorem{prin}[thm]{Principe}



\theoremstyle{definition}
\newtheorem{defn}{Définition}[chapter]
\newtheorem{exmp}{Exemple}[chapter]

\theoremstyle{remark}
\newtheorem{rmk}{Remarque}[chapter]
\newtheorem{rap}[rmk]{Rappel}
\newtheorem{notat}[rmk]{Notation}

\everymath{\displaystyle}

\pagestyle{headings}
\headheight = 12pt
\headsep = 25pt